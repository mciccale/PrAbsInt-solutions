\documentclass[10pt,justified,nofonts]{tufte-handout}
%
% Core -------------------------------------------------------------------------%
\usepackage[english]{babel}
\usepackage{etoolbox}
\usepackage{graphicx}
\usepackage{relsize}
\usepackage{natbib}
%
\setcounter{secnumdepth}{2}

% Math -------------------------------------------------------------------------%
\usepackage{amsmath}
\usepackage{amssymb}
\usepackage{amsthm}
%
\newtheoremstyle{answerstyle} % name
  {10pt}
  {10pt}
  {\normalfont}
  {}
  {\bfseries} % Heading font
  {.}
  {.5em}
  {\thmname{#1}\thmnumber{ #2}\thmnote{ (#3)}} % Heading spec
\theoremstyle{answerstyle}
\newtheorem{exerciseinner}{Exercise}
\newenvironment{exercise}[1]{%
  \IfBlankTF{#1}%
    {\renewcommand{\theexerciseinner}{\unskip}}%
    {\renewcommand\theexerciseinner{#1}}%
  \exerciseinner%
}{\endexerciseinner}
%
\newenvironment{answer}{%
  \renewcommand{\qedsymbol}{\ensuremath{\Box}}
  \proof[Answer]
}{\endproof}

% Hyperref ---------------------------------------------------------------------%
\usepackage[dvipsnames]{xcolor}
%
\definecolor[named]{MyPurple}{cmyk}{0.55,1,0,0.15}
\definecolor[named]{MyDarkBlue}{cmyk}{1,0.58,0,0.21}
\definecolor[named]{keywordcolor}{rgb}{0.7, 0.1, 0.1}
\definecolor[named]{tacticcolor}{rgb}{0.0, 0.1, 0.6}
\definecolor[named]{commentcolor}{rgb}{0.4, 0.4, 0.4}
\definecolor[named]{symbolcolor}{rgb}{0.0, 0.1, 0.6}
\definecolor[named]{sortcolor}{rgb}{0.1, 0.5, 0.1}
\definecolor[named]{attributecolor}{rgb}{0.7, 0.1, 0.1}
%
\usepackage{hyperref}
\hypersetup{%
  pdftitle={Marco Ciccal{\`{e}} --- Solutions to exercises of
    ``Principles of Abstract Interpretation'' by Cousot (2021)},%
  colorlinks=true,%
  linkcolor=MyDarkBlue,%
  citecolor=MyDarkBlue,%
  urlcolor=MyDarkBlue,%
  filecolor=MyDarkBlue,%
  bookmarksnumbered=true,%
  linktocpage=true%
}
%
\usepackage[capitalize,noabbrev]{cleveref}
%
\crefformat{section}{\S#2#1#3}
\Crefformat{section}{\S#2#1#3}
%
\crefformat{equation}{(#2#1#3)}
\Crefformat{equation}{(#2#1#3)}
%
\usepackage{url}

% Fonts -----------------------------------------------------------------------%
\usepackage[T1]{fontenc}
\usepackage[utf8]{inputenc}
\usepackage[tt=false]{libertine}
\usepackage[libertine]{newtxmath}
\usepackage[scaled=.85]{beramono}
\usepackage{BOONDOX-cal}
%
\font\bboldfontten=bbold10 at 10pt
\font\bboldfonteight=bbold10 at 8pt
\font\bboldfontsix=bbold10 at 6pt
\renewcommand{\mathbb}[1]
  {\mathchoice%
    {\mbox{\bboldfontten#1}}%
    {\mbox{\bboldfontten#1}}
    {\mbox{\scriptsize\bboldfonteight#1}}%
    {\mbox{\tiny\bboldfontsix#1}}}

% Listings --------------------------------------------------------------------%
\usepackage{listings}
%
\def\lstlanguagefiles{lstlean.tex}
\lstset{language=lean,aboveskip=\baselineskip,belowskip=0pt}

% Macros ----------------------------------------------------------------------%
\renewcommand{\implies}
  {\ensuremath{\Rightarrow}}
\newcommand{\ttrue}
  {\ensuremath{\textbf{\textsf{tt}}}}
\newcommand{\ffalse}
  {\ensuremath{\textbf{\textsf{ff}}}}
\newcommand{\isDefAs}
  {\ensuremath{\triangleq}}
\newcommand{\Lean}
  {\textsf{Lean4}}
\newcommand{\intset}
  {\ensuremath{\mathbb{Z}}}
\newcommand{\sorry}
  {\textcolor{red}{Sorry!}}

%%% Local Variables:
%%% mode: LaTeX
%%% TeX-master: "solutions"
%%% End:

%
\title{Solutions to exercises of ``Principles of Abstract Interpretation'' by Cousot (2021)}
\author{Marco Ciccal{\`{e}}}
%
\begin{document}
%
\maketitle
%
\tableofcontents
%
\section{Chapter 3}
%
\begin{exercise}{3.6}
  Define the logical operators (negation $\neg$, implication
  $\implies$, conjunction $\wedge$, disjunction $\vee$) in terms of
  the Sheffer stroke $\uparrow$.\footnote{Also known as NAND (``not and'').}
\end{exercise}
%
\begin{answer}
  Let us first recall the definition of the Sheffer stroke $\uparrow$
  in the form of a truth table:
  %
  %
  
  \begin{table}
    \centering
    \caption{Truth table of the Sheffer stroke $\uparrow$.}
    \begin{tabular}{c|c|c|c|c}
      $a$ & $\ttrue$ & $\ttrue$ & $\ffalse$ & $\ffalse$\\\hline
      $b$ & $\ttrue$ & $\ffalse$ & $\ttrue$ & $\ffalse$\\\hline
      $a \uparrow b$ & $\ffalse$ & $\ttrue$ & $\ttrue$ & $\ttrue$
    \end{tabular}
  \end{table}
  %
  %
  
  \noindent We can define the logical operators above as
  follows:\footnote{Out of curiosity, the Sheffer stroke $\uparrow$ is
    ``functionally complete,'' i.e., it can be used to express
    \emph{all} possible truth tables.}
  %
  \begin{align*}
          \neg a &\isDefAs a \uparrow a\tag{$\neg$}\\
    a \implies b &\isDefAs a \uparrow (b \uparrow b)\tag{$\implies$}\\
    a   \wedge b &\isDefAs (a \uparrow b) \uparrow (a \uparrow b)\tag{$\wedge$}\\
    a     \vee b &\isDefAs (a \uparrow a) \uparrow (b \uparrow b)\tag{$\vee$}
  \end{align*}
\end{answer}
%
\begin{exercise}{3.7}
  Write a program in the language of your choice that inputs (an
  encoding of) an expression (with no variables) and returns the value
  of this expression.
\end{exercise}
%
\begin{answer}
  In the following \Lean\ program, we first define a sum type
  \texttt{ExpKind} for discriminating between \emph{arithmetic} and
  \emph{boolean} expressions, together with the
  \texttt{ExpKind.interp} funtion mapping each constructor with its
  corresponding \Lean.\footnote{The \texttt{@[reducible]} directive
    allows us to use the function inside the predicate of a dependent
    type.}
  %
  Next, we represent expressions with the \texttt{Exp} inductive
  data type.
  %
  Finally, we define the \texttt{Exp.eval} function which
  accepts: the kind of the expression to evaluate;\footnote{The curly braces notation
    represents an \emph{implicit} parameter.} the expression itself; and yields the interpretation
  of evaluating such expression.
  %
  \begin{lstlisting}
  inductive ExpKind where
    | arith
    | bool
  
  @[reducible]
  def ExpKind.interp : ExpKind → Type
    | arith => Int
    | bool  => Bool
  
  inductive Exp : ExpKind → Type where
    | one  : Exp .arith
    | sub  : Exp .arith → Exp .arith → Exp .arith
    | tt   : Exp .bool
    | ff   : Exp .bool
    | lt   : Exp .arith → Exp .arith → Exp .bool
    | nand : Exp .bool → Exp .bool → Exp .bool
  
  def Exp.eval {kind: ExpKind} : Exp kind → kind.interp
    | .one        => 1
    | .sub  e₁ e₂ => eval e₁ - eval e₂
    | .tt         => true
    | .ff         => false
    | .lt   e₁ e₂ => eval e₁ < eval e₂
    | .nand e₁ e₂ => !(eval e₁ && eval e₂)
  \end{lstlisting}
  %
  %
  
  For example:
  %
  \begin{lstlisting}
  #eval Exp.eval (Exp.sub (Exp.sub Exp.one Exp.one) Exp.one)
  \end{lstlisting}
  %
  yields \texttt{-1}; and
  %
  \begin{lstlisting}
  #eval Exp.eval (Exp.nand (Exp.lt Exp.one Exp.one) Exp.tt)
  \end{lstlisting}
  %
  yields \texttt{true}.\marginnote{The source code is available under
    the \texttt{Lean4Solutions} \Lean\ project, in the
    \href{file:./Lean4Solutions/Lean4Solutions/Chapter3/Exercise3-7.lean}{\tt
      Exercise3-7.lean} module.}
  %
\end{answer}
%
\begin{exercise}{3.8}
  Prove that any integer $z \in \intset$ has a finite denotation in
  the syntax $\mathbb{A}$ of arithmetic expressions.
\end{exercise}
%
\begin{proof}
  The proof proceeds by (complete) induction over $\intset$.
  %
  \begin{itemize}
  \item Base case ($z = 0$).
    %
    The (finite) expression \texttt{1 - 1} denotes the integer $0$
    under the semantics of arithmetic expressions, i.e.,
    $\mathbcal{A}\mathbb{[}\texttt{E}\mathbb{]} = 0$.
  %
  \item Inductive step ($z \leqslant 0$).
    %
    Assume that for $z \leqslant 0$ there exists an expression
    \texttt{E}$_z$
    s.t.~$\mathbcal{A}\mathbb{[}\texttt{E}_z\mathbb{]} = z$.
    %
    We define the expression
    \texttt{E}$_{z-1} \isDefAs \texttt{E}_z\texttt{ - 1}$,
    whose semantics is:
    %
    \begin{align*}
      \mathbcal{A}\mathbb{[}\texttt{E}_{z-1}\mathbb{]}
      &= \mathbcal{A}\mathbb{[}\texttt{E}_z\mathbb{]} - \mathbcal{A}\mathbb{[}\texttt{1}\mathbb{]}\\
      &= z - 1
    \end{align*}
  %
  \item Inductive step ($z \geqslant 0$).
    %
    Assume that for $z \geqslant 0$ there exists an expression
    \texttt{E}$_z$
    s.t.~$\mathbcal{A}\mathbb{[}\texttt{E}_z\mathbb{]} = z$.
    %
    We define the expression
    \texttt{E}$_{z+1} \isDefAs \texttt{E}_z\texttt{ - ((1 - 1) - 1)}$,
    whose semantics is:
    %
    \begin{align*}
      \mathbcal{A}\mathbb{[}\texttt{E}_{z+1}\mathbb{]}
      &= \mathbcal{A}\mathbb{[}\texttt{E}_z\mathbb{]} - \mathbcal{A}\mathbb{[}\texttt{((1 - 1) - 1)}\mathbb{]}\\
      &= \mathbcal{A}\mathbb{[}\texttt{E}_z\mathbb{]} - (\mathbcal{A}\mathbb{[}\texttt{(1 - 1)}\mathbb{]} - \mathbcal{A}\mathbb{[}\texttt{1}\mathbb{]})\\
      &= \mathbcal{A}\mathbb{[}\texttt{E}_z\mathbb{]} - ((\mathbcal{A}\mathbb{[}\texttt{1}\mathbb{]} - \mathbcal{A}\mathbb{[}\texttt{1}\mathbb{]}) - \mathbcal{A}\mathbb{[}\texttt{1}\mathbb{]})\\
      &= z - ((1 - 1) - 1)\\
      &= z + 1\qedhere
    \end{align*}
  \end{itemize}
\end{proof}
%
\begin{exercise}{3.9}
  Let $\iota \in \mathbb{V} \to \mathbb{V}$ be an isomorphism (i.e., a
  renaming of variables).
  %
  Define $\iota\mathbb{[}\texttt{E}\mathbb{]}$ to be the expression obtained
  by renaming of the variables \texttt{x} of \texttt{E} to $\iota(x)$.
  %
  Prove that
  $\mathbcal{S}\mathbb{[}\texttt{E}\mathbb{]}\rho =
  \mathbcal{S}\mathbb{[}\iota\mathbb{[}\texttt{E}\mathbb{]}\mathbb{]}(\texttt{x}
  \in \mathbb{V} \mapsto \rho(\iota(\texttt{x})))$.
  %
  This show, intuitively, that the renaming of variables does not
  change the semantics of expressions.
\end{exercise}
%
\begin{answer}
  We first define the $\iota\mathbb{[}\texttt{E}\mathbb{]}$
  isomorphism as follows:
  %
  \begin{align*}
    \iota\mathbb{[}\texttt{1}\mathbb{]} &\isDefAs \texttt{1}\\
    \iota\mathbb{[}\texttt{x}\mathbb{]} &\isDefAs \iota(\texttt{x})\\
    \iota\mathbb{[}\texttt{A}_1\texttt{ - }\texttt{A}_2\mathbb{]} &\isDefAs \iota\mathbb{[}\texttt{A}_1\mathbb{]}\texttt{ - }\iota\mathbb{[}\texttt{A}_2\mathbb{]}\\
    \iota\mathbb{[}\texttt{A}_1\texttt{ < }\texttt{A}_2\mathbb{]} &\isDefAs \iota\mathbb{[}\texttt{A}_1\mathbb{]}\texttt{ < }\iota\mathbb{[}\texttt{A}_2\mathbb{]}\\
    \iota\mathbb{[}\texttt{B}_1\texttt{ nand }\texttt{B}_2\mathbb{]} &\isDefAs \iota\mathbb{[}\texttt{B}_1\mathbb{]}\texttt{ nand }\iota\mathbb{[}\texttt{B}_2\mathbb{]}
  \end{align*}
  %
  %
  
  Next, let us prove that the semantics of expressions is not altered by
  variable renaming.
  %
  \begin{proof}
    We proceed by structural induction on the size of the expression \texttt{E}.
    %
    \begin{itemize}
    \item Base case (\texttt{E} $\equiv$ \texttt{1}).
    \item Base case (\texttt{E} $\equiv$ \texttt{x}).
    \item Inductive step (\texttt{E} $\equiv$ \texttt{A}$_1$\texttt{ - A}$_2$).
    \item Inductive step (\texttt{E} $\equiv$ \texttt{A}$_1$\texttt{ < A}$_2$).
    \item Inductive step (\texttt{E} $\equiv$ \texttt{B}$_1$\texttt{ nand B}$_2$).
    \end{itemize}
  \end{proof}
\end{answer}
%
\phantomsection
\bibliographystyle{tlplike}
\bibliography{biblio.bib}
\end{document}

%%% Local Variables:
%%% mode: LaTeX
%%% TeX-master: t
%%% End:
