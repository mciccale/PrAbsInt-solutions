\documentclass[10pt,justified,nofonts]{tufte-handout}
%
\usepackage[english]{babel}
\usepackage{etoolbox}
\usepackage{graphicx}
\usepackage{relsize}
\usepackage{natbib}
%
\setcounter{secnumdepth}{2}
%
\usepackage{amsmath}
\usepackage{amssymb}
\usepackage{amsthm}
%
\newtheorem{exerciseinner}{Exercise}
\newenvironment{exercise}[1]{%
  \IfBlankTF{#1}
    {\renewcommand{\theexerciseinner}{\unskip}}
    {\renewcommand\theexerciseinner{#1}}%
  \exerciseinner
}{\endexerciseinner}
%
\newenvironment{answer}[0]{%
  \par\pushQED{\qed}%
  \renewcommand\qedsymbol{$\Box$}% You can change this symbol
  \noindent\textit{Answer.}%
  \hspace{.5em}%
}{\popQED\par}
%
\usepackage{natbib}
%
\usepackage[dvipsnames]{xcolor}
%
\definecolor[named]{MyPurple}{cmyk}{0.55,1,0,0.15}
\definecolor[named]{MyDarkBlue}{cmyk}{1,0.58,0,0.21}
%
\usepackage{hyperref}
\hypersetup{%
  pdfborder={0 0 0},%
  pdftitle={Marco Ciccal{\`{e}} --- Solutions to exercises of
    ``Principles of Abstract Interpretation'' by Cousot (2021)},%
  colorlinks=true,%
  linkcolor=MyDarkBlue,%
  citecolor=MyDarkBlue,%
  urlcolor=MyDarkBlue,%
  filecolor=MyDarkBlue,%
  bookmarksnumbered=true,%
  linktocpage=true%
}
%
\usepackage[capitalize,noabbrev]{cleveref}
%
\crefformat{section}{\S#2#1#3}
\Crefformat{section}{\S#2#1#3}
%
\crefformat{equation}{(#2#1#3)}
\Crefformat{equation}{(#2#1#3)}
%
\usepackage{url}
%
\usepackage[T1]{fontenc}
\usepackage[utf8]{inputenc}
\usepackage[tt=false,osf=true]{libertine}
\usepackage[libertine]{newtxmath}
\usepackage[scaled=.85]{beramono}
\usepackage{BOONDOX-cal}
%
\usepackage{listings}
%
\definecolor{keywordcolor}{rgb}{0.7, 0.1, 0.1}   % red
\definecolor{tacticcolor}{rgb}{0.0, 0.1, 0.6}    % blue
\definecolor{commentcolor}{rgb}{0.4, 0.4, 0.4}   % grey
\definecolor{symbolcolor}{rgb}{0.0, 0.1, 0.6}    % blue
\definecolor{sortcolor}{rgb}{0.1, 0.5, 0.1}      % green
\definecolor{attributecolor}{rgb}{0.7, 0.1, 0.1} % red

\def\lstlanguagefiles{lstlean.tex}
% set default language
\lstset{language=lean,aboveskip=\baselineskip,belowskip=0pt}
%
\font\bboldfontten=bbold10 at 10pt
\font\bboldfonteight=bbold10 at 8pt
\font\bboldfontsix=bbold10 at 6pt
\renewcommand{\mathbb}[1]
  {\mathchoice%
    {\mbox{\bboldfontten#1}}%
    {\mbox{\bboldfontten#1}}
    {\mbox{\scriptsize\bboldfonteight#1}}%
    {\mbox{\tiny\bboldfontsix#1}}}
%
\renewcommand{\implies}
  {\ensuremath{\Rightarrow}}
\newcommand{\ttrue}
  {\ensuremath{\textbf{\textsf{tt}}}}
\newcommand{\ffalse}
  {\ensuremath{\textbf{\textsf{ff}}}}
\newcommand{\isDefAs}
  {\ensuremath{\triangleq}}
%
\title{Solutions to exercises of ``Principles of Abstract Interpretation'' by Cousot (2021)}
\author{Marco Ciccal{\`{e}}}
%
\begin{document}
%
\maketitle
%
\tableofcontents
%
\section{Solutions to exercises in Chapter 3}
%
\begin{exercise}{3.6}
  Define the logical operators (negation $\neg$, implication
  $\implies$, conjunction $\wedge$, disjunction $\vee$) in terms of
  the Sheffer stroke $\uparrow$.\footnote{Also known as NAND (``not and'').}
\end{exercise}
%
\begin{answer}
  Let us first recall the definition of the Sheffer stroke $\uparrow$
  in the form of a truth table:
  %
  %
  
  \begin{table}
    \centering
    \caption{Truth table of the Sheffer stroke $\uparrow$.}
    \begin{tabular}{c|c|c|c|c}
      $a$ & $\ttrue$ & $\ttrue$ & $\ffalse$ & $\ffalse$\\\hline
      $b$ & $\ttrue$ & $\ffalse$ & $\ttrue$ & $\ffalse$\\\hline
      $a \uparrow b$ & $\ffalse$ & $\ttrue$ & $\ttrue$ & $\ttrue$
    \end{tabular}
  \end{table}
  %
  %
  
  \noindent We can define the logical operators above as
  follows:\footnote{Note that the Sheffer stroke $\uparrow$ is
    ``functionally complete,'' i.e., it can be used to express
    \emph{all} possible truth tables.}
  %
  \begin{align*}
          \neg a &\isDefAs a \uparrow a\tag{$\neg$}\\
    a \implies b &\isDefAs a \uparrow (b \uparrow b)\tag{$\implies$}\\
    a   \wedge b &\isDefAs (a \uparrow b) \uparrow (a \uparrow b)\tag{$\wedge$}\\
    a     \vee b &\isDefAs (a \uparrow a) \uparrow (b \uparrow b)\tag{$\vee$}
  \end{align*}
\end{answer}
%
\begin{exercise}{3.7}
  Write a program in the language of your choice that inputs (an
  encoding of) an expression (with no variables) and returns the value
  of this expression.
\end{exercise}
%
\begin{answer}
  In the following $\mathsf{Lean4}$ program, we first define a sum
  type \texttt{ExpKind} for discriminating between \emph{arithmetic}
  and \emph{boolean} expressions, together with the
  \texttt{ExpKind.interp} funtion mapping each constructor with its
  corresponding $\mathsf{Lean4}$.\footnote{The \texttt{@[reducible]}
    directive allows us to use the function inside a dependent type.}
  %
  Next, we represent expressions with the \texttt{Exp} inductive
  data type.
  %
  Finally, we define the \texttt{Exp.eval} function which
  accepts: the kind of the expression to evaluate;\footnote{The curly braces notation
    represents an \emph{implicit} parameter.} the expression itself; and yields the interpretation
  of evaluating such expression.
  %
  \begin{lstlisting}
  inductive ExpKind where
    | arith
    | bool
  
  @[reducible]
  def ExpKind.interp : ExpKind → Type
    | arith => Int
    | bool  => Bool
  
  inductive Exp : ExpKind → Type where
    | one  : Exp .arith
    | sub  : Exp .arith → Exp .arith → Exp .arith
    | tt   : Exp .bool
    | ff   : Exp .bool
    | lt   : Exp .arith → Exp .arith → Exp .bool
    | nand : Exp .bool → Exp .bool → Exp .bool
  
  def Exp.eval {kind: ExpKind} : Exp kind → kind.interp
    | .one        => 1
    | .sub  e₁ e₂ => eval e₁ - eval e₂
    | .tt         => true
    | .ff         => false
    | .lt   e₁ e₂ => eval e₁ < eval e₂
    | .nand e₁ e₂ => !(eval e₁ && eval e₂)
  \end{lstlisting}
  %
  For example:
  %
  \begin{lstlisting}
  #eval Exp.eval (Exp.sub (Exp.sub Exp.one Exp.one) Exp.one)
  \end{lstlisting}
  %
  yields \texttt{-1}; and
  %
  \begin{lstlisting}
  #eval Exp.eval (Exp.nand (Exp.lt Exp.one Exp.one) Exp.tt)
  \end{lstlisting}
  %
  yields \texttt{true}.\marginnote{The source code is available under
    the \texttt{Lean4Solutions} $\mathsf{Lean4}$ project, in the
    \href{file:./Lean4Solutions/Lean4Solutions/Chapter3/Exercise3-7.lean}{\tt
      Exercise3-7.lean} module.}
  %
\end{answer}
%
\phantomsection
\bibliographystyle{tlplike}
\bibliography{biblio.bib}
\end{document}

%%% Local Variables:
%%% mode: LaTeX
%%% TeX-master: t
%%% End:
